\documentclass[crop]{standalone}
\usepackage{tikz}
\begin{document}
\begin{tikzpicture}
  \draw (0,0) rectangle (12,6);
  \node[align=center] at (6,3) {SD-AQI Timeseries\\(placeholder figure)};
\end{tikzpicture}
\end{document}
